\documentclass{article}
\usepackage[margin = .7in]{geometry}
\usepackage[dvipdfmx]{graphicx}
\usepackage{listings}
\usepackage{amsmath}
\usepackage{bm}
\lstset{%
  language={python},
  basicstyle={\small},%
  identifierstyle={\small},%
  commentstyle={\small\itshape},%
  keywordstyle={\small\bfseries},%
  ndkeywordstyle={\small},%
  stringstyle={\small\ttfamily},
  frame={tb},
  breaklines=true,
  columns=[l]{fullflexible},%
  numbers=left,%
  xrightmargin=0zw,%
  xleftmargin=3zw,%
  numberstyle={\scriptsize},%
  stepnumber=1,
  numbersep=1zw,%
  lineskip=-0.5ex%
}

\begin{document}
\title{STAT6011/7611/6111/3317 \\ 
COMPUTATIONAL STATISTICS (2016 Fall) \\
Assignment 4}
\author{Kei Ikegami (u3535947)}
\maketitle

\section{}
The code is below. I denote the 'Fair' by 0 and 'Loaded' by 1 in this code.
	\lstinputlisting[caption=problem 1]{1.py}
And the results are as follows.
\begin{description}
	\item[(a)] [ 0,  0,  0,  0,  0,  0,  0,  0,  0,  0,  0,  0,  0,
      	0,  0,  0,  0,  0,  1,  1,  1,  1,  1,  1,  1,  1,
        0,  0,  1,  1,  1,  1,  0,  0,  1,  1,  1,  1,  1,
        0,  0,  0,  0,  0,  0,  0,  0,  0,  0,  0,  0,  0,
        0,  0,  0,  0,  0,  0,  0,  0,  0,  0,  0,  0,  0,
        0,  0,  0,  0]
	\item[(b)] [ 0,  0,  0,  0,  0,  0,  0,  0,  0,  0,  0,  0,  0,
        0,  0,  0,  0,  0,  0,  1,  1,  1,  1,  1,  1,  1,
        1,  1,  1,  1,  1,  1,  1,  1,  1,  1,  1,  1,  1,
        1,  0,  1,  0,  0,  0,  0,  0,  0,  0,  0,  0,  0,
        0,  0,  0,  0,  0,  0,  0,  0,  0,  0,  0,  0,  0,
        0,  0,  0,  0]
	\item[(c)] [ 0,  1,  0,  0,  0,  1,  0,  0,  0,  0,  0,  0,  0,
        1,  0,  0,  1,  1,  1,  1,  1,  1,  0,  1,  1,  0,
        0,  1,  1,  1,  1,  0,  0,  1,  1,  1,  1,  1,  0,
        0,  0,  1,  0,  0,  1,  1,  0,  0,  0,  0,  0,  0,
        0,  1,  0,  0,  0,  0,  1,  0,  0,  1,  0,  0,  0,
        1,  0,  0,  0]
	\item[(d)] [ 0,  1,  0,  0,  0,  1,  0,  0,  0,  0,  0,  0,  0,
        1,  0,  0,  1,  1,  1,  1,  1,  1,  0,  1,  1,  0,
        0,  1,  1,  1,  1,  0,  0,  1,  1,  1,  1,  1,  0,
        0,  0,  1,  0,  0,  1,  1,  0,  0,  0,  0,  0,  0,
        0,  1,  0,  0,  0,  0,  1,  0,  0,  1,  0,  0,  0,
        1,  0,  0,  0]
\end{description}
\section{}
	\lstinputlisting[caption=problem 2]{2.py}
	In the below question, I mean the weights by the values multiplied by weight functions.
	\subsection{}
	 	I prove the integral of standard normal probability distribution function is equal to $1$.
	\subsection{}
		I use the above code to compute the approximation of the integral.\par
		When the number of nodes is 5, \par the approximation result is 0.99684628222456162, \par the nodes are [-2.02018287, -0.95857246,  0.        ,  0.95857246,  2.02018287], \par the weights are [1.1814886255359833,
 0.98658099675142807,
 0.94530872048294179,
 0.98658099675142807,\par
 1.1814886255359833]
 		\par
		\ 
		\par
		When the number of nodes is 10, \par
		the approximation result is 0.99998763906433263 \par
		the nodes are [-3.43615912, -2.53273167, -1.75668365, -1.03661083, -0.34290133,
         0.34290133,  1.03661083,  1.75668365, \par 2.53273167,  3.43615912], \par
         	and the weights are [1.0254516913657519,
 0.82066612640481784,
 0.74144193194356545,
 0.70329632310490586,\par
 0.6870818539512733,
 0.6870818539512733,
 0.70329632310490586,
 0.74144193194356545,
 0.82066612640481784,\par
 1.0254516913657519] \par
 		\ 
		\par
 		When the number of nodes is 20, \par
		the approximation result is 0.99999999979803234, \par
		the nodes are [-5.38748089, -4.60368245, -3.94476404, -3.34785457, -2.78880606,
        -2.254974  , -1.73853771, -1.23407622, \par -0.73747373, -0.24534071,
         0.24534071,  0.73747373,  1.23407622,  1.73853771,  2.254974  ,
         2.78880606,  3.34785457,  \par 3.94476404,  4.60368245,  5.38748089], \par
         	and the weights are [0.89859196145317,
 0.70433296117692357,
 0.62227869619138665,
 0.57526244285250083,\par
 0.54485174236452072,
 0.52408035094855054,
 0.50967902711745705,
 0.49992087133628998,
 0.4938433852720529,\par
 0.49092150066674595,
 0.49092150066674595,
 0.4938433852720529,
 0.49992087133628998,
 0.50967902711745705,\par
 0.52408035094855054,
 0.54485174236452072,
 0.57526244285250083,
 0.62227869619138665,
 0.70433296117692357,\par
 0.89859196145317] \par
 		\ 
		\par
 		When the number of nodes is 30, \par
		the approximation result is 0.99999999999999634, \par
		the nodes are [-6.86334529, -6.13827922, -5.53314715, -4.98891897, -4.48305536,
        -4.0039086 , -3.54444387, -3.09997053, \par-2.66713212, -2.24339147,
        -1.82674114, -1.4155278 , -1.00833827, -0.60392106, -0.20112858,
         0.20112858,  0.60392106, \par 1.00833827,  1.4155278 ,  1.82674114,
         2.24339147,  2.66713212,  3.09997053,  3.54444387,  4.0039086 ,
         4.48305536,  \par 4.98891897,  5.53314715,  6.13827922,  6.86334529], \par
         and the weights are [0.83424747101269592,
 0.64909798155433118,
 0.56940269194957616,
 0.52252568933130883,\par
 0.49105799583287552,
 0.4683748125647248,
 0.451321035991189,
 0.43817702265268194,
 0.42791806293273793,\par
 0.41989500373682354,
 0.41367936361113872,
 0.40898157500353133,
 0.40560512332568432,
 0.40341981692480389,\par
 0.40234606670190304,
 0.40234606670190304,
 0.40341981692480389,
 0.40560512332568432,
 0.40898157500353133,\par
 0.41367936361113872,
 0.41989500373682354,
 0.42791806293273793,
 0.43817702265268194,
 0.451321035991189,\par
 0.4683748125647248,
 0.49105799583287552,
 0.52252568933130883,
 0.56940269194957616,
 0.64909798155433118,\par
 0.83424747101269592]. \par
         
\section{}
	\lstinputlisting[caption=problem 3]{3.py}
	In the below questions, I mean the weights by the values not multiplied by the weight functions.
	\subsection{}
		When I use Legendre polynomial as orthogonal polynomial, 
		\par
		the approximation result is -49.506283813990549,
		\par
		the nodes are [-0.90617985, -0.53846931,  0.        ,  0.53846931,  0.90617985],\par
		the weights are [ 0.23692689,  0.47862867,  0.56888889,  0.47862867,  0.23692689].
		\par
		\ 
		\par
		When I use Chebyshev type 1 polynomial as orthogonal polynomial, 
		\par
		the approximation result is -57.161045874594777,
		\par
		the nodes are [ -9.51056516e-01,  -5.87785252e-01,   6.12323400e-17,
          5.87785252e-01,   9.51056516e-01,]\par
		the weights are [ 0.62831853,  0.62831853,  0.62831853,  0.62831853,  0.62831853].
		\par
		\ 
		\par
		When I use Chebyshev type 2 polynomial as orthogonal polynomial, 
		\par
		the approximation result is -40.789633611622008,
		\par
		the nodes are [ -8.66025404e-01,  -5.00000000e-01,   6.12323400e-17,
          5.00000000e-01,   8.66025404e-01]\par
		the weights are [ 0.13089969,  0.39269908,  0.52359878,  0.39269908,  0.13089969].
		
	\subsection{}
		When I use Legendre polynomial as orthogonal polynomial, 
		\par
		the approximation result is -49.493963006199031,
		\par
		the nodes are [-0.97390653, -0.86506337, -0.67940957, -0.43339539, -0.14887434,
         0.14887434,  0.43339539,  0.67940957, \par 0.86506337,  0.97390653],\par
		the weights are [ 0.06667134,  0.14945135,  0.21908636,  0.26926672,  0.29552422,
         0.29552422,  0.26926672,  0.21908636, \par 0.14945135,  0.06667134].
		\par
		\ 
		\par
		When I use Chebyshev type 1 polynomial as orthogonal polynomial, 
		\par
		the approximation result is -50.987455239343021,
		\par
		the nodes are [-0.98768834, -0.89100652, -0.70710678, -0.4539905 , -0.15643447,
         0.15643447,  0.4539905 ,  0.70710678, \par 0.89100652,  0.98768834]\par
		the weights are [ 0.31415927,  0.31415927,  0.31415927,  0.31415927,  0.31415927,
         0.31415927,  0.31415927,  0.31415927, \par 0.31415927,  0.31415927].
		\par
		\ 
		\par
		When I use Chebyshev type 2 polynomial as orthogonal polynomial, 
		\par
		the approximation result is -47.101342989695162,
		\par
		the nodes are [-0.95949297, -0.84125353, -0.65486073, -0.41541501, -0.14231484,
         0.14231484,  0.41541501,  0.65486073, \par 0.84125353,  0.95949297]\par
		the weights are [ 0.02266894,  0.08347854,  0.16312218,  0.23631356,  0.27981494,
         0.27981494,  0.23631356,  0.16312218, \par 0.08347854,  0.02266894].
\section{}
	\lstinputlisting[caption=problem 4]{4.py}
	\subsection{}
		Let $\theta = (p_a, p_b, p_o)$ and ${\bf Y_{obs}} = (n_A, n_B, n_O, n_{AB})$. The missing values are ${\bf Z} = (z_{AO}, z_{BO})$. I calculate Q-function as follows.
		\begin{align}
			Q(\theta | \theta^{(t)}) &= E_{\theta^{(t)}} [{\rm log} L(\theta | {\bf Y_{obs}}, {\bf Z}) | \theta^{(t)}, {\bf Y_{obs}}]
			=  \int {\rm log} L(\theta | {\bf Y_{obs}}, {\bf Z}) f({\bf Z} | \theta^{(t)}, {\bf Y_{obs}}) \mathrm{d} {\bf Z} \nonumber \\
			&= \sum_{j = 0}^{n_{B}} \sum_{i = 0}^{n_{A}} {\rm log} L(\theta | {\bf Y_{obs}}, z_{AO} = i, z_{BO} = j) P(z_{AO} = i, z_{BO} = j | \theta^{(t)}, {\bf Y_{obs}})
		\end{align}
		Now I can get the joint probability mass function of ($z_{AO}, z_{BO}$) as follows.
		\begin{align*}
			P_{ij} &= P(z_{AO} = i, z_{BO} = j | \theta^{(t)}, {\bf Y_{obs}}) \\
			&= \binom{n_A}{i} (q_A)^i (1-q_A)^{n_A - i} *  \binom{n_B}{j} (q_B)^j (1-q_B)^{n_B - j}
		\end{align*}
		where
		\begin{align*}
			q_A = \frac{2p_A^{(t)} p_O^{(t)}}{2p_A^{(t)} p_O^{(t)} + (p_A^{(t)})^2} = \frac{2p_O^{(t)}}{2p_O^{(t)} + p_A^{(t)}} \\[10pt]
			q_B = \frac{2p_B^{(t)} p_O^{(t)}}{2p_B^{(t)} p_O^{(t)} + (p_B^{(t)})^2} = \frac{2p_O^{(t)}}{2p_O^{(t)} + p_B^{(t)}}
		\end{align*}
		And the log likelihood function is transformed into the below.
		\begin{align*}
			{\rm log} L(\theta | {\bf Y_{obs}}, z_{AO} = i, z_{BO} = j) &= {\rm log} (2p_A p_O)^i (p_A^2)^{n_A - i} (2p_Bp_O)^j (p_B^2)^{n_B - j} (p_O^2)^{n_O} (2p_A p_B)^{n_{AB}}\\
			&= (2n_A + n_{AB} - i) {\rm log}\ p_A + (2n_B + n_{AB} - j) {\rm log}\ p_B \\&\quad + (2n_O + i + j) {\rm log}\ p_O + \text{unrelated terms}
		\end{align*}
		Then the maximization step is written as follows.
		\begin{align*}
			&\max_{\theta} \sum_{j = 0}^{n_{B}} \sum_{i = 0}^{n_{A}} P_{ij} (2n_A + n_{AB} - i) {\rm log}\ p_A + (2n_B + n_{AB} - j) {\rm log}\ p_B + (2n_O + i + j) {\rm log}\ p_O \\
			&\text{s.t.}\ p_A + p_B + p_O = 1
		\end{align*}
		By using the method of Lagrange multiplier, we can get the below conditions.
		\begin{align*}
			\frac{\partial Q}{\partial p_A} = \frac{\sum_j \sum_i P_{ij} (2n_A + n_{AB} -i)}{p_A} - \frac{\sum_j \sum_i P_{ij} (2n_O + i + j)}{1 - p_A - p_B} = 0 \\[10pt]
			\frac{\partial Q}{\partial p_B} = \frac{\sum_j \sum_i P_{ij} (2n_B + n_{AB} -i)}{p_B} - \frac{\sum_j \sum_i P_{ij} (2n_O + i + j)}{1 - p_A - p_B} = 0
		\end{align*}
		Since $n_A = n_B$, the above two conditions mean that $p_A^{(t +1)} = p_B^{(t+1)}$. Thus I get the maximizer of Q function under the constraint.
		\begin{align*}
			&p_B^{(t+1)} = p_A^{(t+1)} = \frac{1}{2 + \frac{\sum_j \sum_i P_{ij} (2n_O + i + j)}{\sum_j \sum_i P_{ij} (2n_A + n_{AB} - i)}}\\[10pt]
			&p_O^{(t+1)} = 1 - 2 p_A^{(t+1)}
		\end{align*}
		Note that $P_{ij}$ depends on the previous step estimation results.
		Repeat the above improvement until converge.
	\subsection{}
	By using the above code, I get the estimation result after 100 iterations. \par The result is $[p_a, p_b, p_o] = [ 0.28021178,  0.28021178,  0.43957643]$.
\section{}
	\subsection{}
		First I specify the distribution of the missing values, which is $u_i$, conditional on the observed values.
		\begin{align*}
			f(u_i | y_i, \theta) &\propto f(u_i, y_i) = f(y_i | u_i) f(u_i) = f(y_{i, 1}, y_{i, 2}, \dots, y_{i, J} | u_i) f(u_i) = \left\{ \Pi_j f(y_{i, j} | u_i)\right\} f(u_i)\\[10pt]
			&\propto \left\{ \Pi_j \exp(-\frac{(y_{ij} - \beta_0 -\beta_1 x_{ij} - u_i)^2}{2\sigma_{\epsilon}^2}) \right\} \exp(-\frac{u_i^2}{2\sigma_u^2})\\[10pt]
			&\propto \exp \left(-\frac{1}{2\sigma_\epsilon^2} \left( \sum_j u_i^2 - 2u_i \sum_j (y_{ij} - \beta_0 - \beta_1 x_{ij}) \right) - \frac{u_i^2}{2\sigma_u^2} \right)\\[10pt]
			&\propto \exp \left( -\frac{J\sigma_u^2 + \sigma_\epsilon^2}{2\sigma_u^2 \sigma_\epsilon^2} \left( u_i - \frac{\sigma_u^2}{J\sigma_u^2 + \sigma_\epsilon^2} \sum_j (y_{ij} - \beta_0 - \beta_1 x_{ij}) \right)^2 \right)
		\end{align*}
		This means that $f(u_i | y_i , \theta) = N\left( \frac{\sigma_u^2}{J\sigma_u^2 + \sigma_\epsilon^2} \sum_j (y_{ij} - \beta_0 - \beta_1 x_{ij}), \frac{\sigma_u^2 \sigma_\epsilon^2}{J\sigma_u^2 + \sigma_\epsilon^2} \right)$. Let this mean be $E_t$ and the variance be $V_t$ when the parameters are t-step estimation results.\par
		Then I get Q function as follows.
		\begin{align}
			Q(\theta | \theta^{(t)}) &= \int \dots \int {\rm log} L(\theta | y, u) \Pi_i f(u_i | y_i, \theta^{(t)}) \mathrm(d)u_1 \dots \mathrm(d)u_I \nonumber \\
			&= \int \dots \int \left\{ \sum_i \sum_j {\rm log} f(y_{ij} | u_i) + {\rm log} f(u_i) \right\} \Pi_i f(u_i | y_i, \theta^{(t)}) \mathrm(d)u_1 \dots \mathrm(d)u_I \nonumber \\
			&= \sum_i \left[ \int \left\{ \sum_j {\rm log} f(y_{ij} | u_i) + {\rm log} f(u_i) \right\} f(u_i | y_i) \mathrm(d)u_i \right] \nonumber \\
			&= \sum_i \sum_j \int {\rm log} f(y_{ij} | u_i) f(u_i | \theta^{(t)}, y_i) \mathrm(d)u_i + \sum_i \sum_j \int {\rm log} f(u_i) f(u_i | \theta^{(t)}, y_i) \mathrm(d)u_i 
		\end{align}
		Then I remove the unrelated terms from (2) and separate it into the left part and the right part. Let the left part of (2) be a, and the right one be b. Now I get the below transformations.
		\begin{align}
			a &= -\sum_i \sum_j \int {\rm log}\ \sigma_\epsilon^2 f(u_i | \theta^{(t)}, y_i) \mathrm(d) u_i - \sum_i \sum_j \int \frac{(y_{ij} - \beta_0 -\beta_1 x_{ij} - u_i)^2}{2\sigma_\epsilon^2} f(u_i | \theta^{(t+1)}, y_i) \mathrm(d)u_i \nonumber \\
			&= -IJ{\rm log}\ \sigma_\epsilon^2 -\frac{\sum_i \sum_j (y_{ij} - \beta - \beta_1 x_{ij})^2}{2\sigma_\epsilon^2} + \frac{\sum_i \sum_j (y_{ij} - \beta - \beta_1 x_{ij}) E_t}{\sigma_\epsilon^2} - \frac{\sum_i \sum_j V_t + E_t^2}{2\sigma_\epsilon^2}\\[15pt]
			b &= -\sum_i \sum_j \int {\rm log}\ \sigma_u^2 f(u_i | \theta^{(t)}, y_i) \mathrm(d)u_i  - \sum_i \sum_j \int \frac{u_i^2}{2\sigma_u^2} f(u_i | \theta^{(t)}, y_i) \mathrm(d)u_i \nonumber \\
			&= -IJ {\rm log}\ \sigma_u^2 -\frac{\sum_i \sum_j V_t + E_t^2}{2\sigma_u^2}
		\end{align}
		By the above, Q function is $a + b$. Then the maximization step is done by differentiating the Q function by each parameter. By using (3) and (4), I get the belows.
		\begin{align}
			&\frac{\partial Q}{\partial \beta_0} = -\frac{IJ}{\sigma_\epsilon^2}\beta_0 + \frac{\sum_i \sum_j (y_{ij} - \beta_1 x_{ij})}{\sigma_\epsilon^2} - \frac{J\sum_i E_t}{\sigma_\epsilon^2} = 0\\[10pt]
			&\frac{\partial Q}{\partial \beta_1} = -\frac{\sum_i \sum_j x_{ij}^2}{\sigma_\epsilon^2}\beta_1 + \frac{\sum_i \sum_j x_{ij} (y_{ij} - \beta_0)}{\sigma_\epsilon^2} - \frac{\sum_i \sum_j x_{ij} E_t}{\sigma_\epsilon^2} = 0\\[10pt]
			&\frac{\partial Q}{\partial \sigma_\epsilon^2} = -\frac{IJ}{2\sigma_\epsilon^2} + \frac{\sum_i \sum_j (y_{ij} - \beta_0 -\beta_1 x_{ij})^2}{2(\sigma_\epsilon^2)^2} - \frac{\sum_i \sum_j (y_{ij} - \beta_0 -\beta_1 x_{ij}) E_t}{(\sigma_\epsilon^2)^2} + \frac{\sum_i \sum_j V_t + E_t^2}{2(\sigma_\epsilon^2)^2} = 0\\[10pt]
			&\frac{\partial Q}{\partial \sigma_u^2} = -\frac{IJ}{2\sigma_u^2} + \frac{\sum_i \sum_j V_t + E_t^2}{2(\sigma_u^2)^2} = 0
		\end{align}
		Note that $E_t$ and $V_t$ depend on the previous estimation results.\par
		By (5) and (6) I can get the next estimations of $\beta_0$ and $\beta_1$ as follows. Let
		\begin{align*}
			\sum_i \sum_j x_{ij} y_{ij} = XY\\
			\sum_i \sum_j x_{ij} = X\\
			\sum_i \sum_j y_{ij} = Y\\
			\sum_i \sum_j x_{ij}^2 = X_2
		\end{align*}
		According to the above notations, 
		\begin{align*}
			&\beta_1^{(t+1)} = \frac{IJ*XY - X*Y - J\left\{ I\sum_i \sum_j x_{ij}E_t - (\sum_i E_t)*X \right\}}{IJ*X_2 - X*X}\\[10pt]
			&\beta_0^{(t+1)} = \frac{Y - \beta_1^{(t)}*X - J \sum_i E_t}{IJ}
		\end{align*}
		By (7) and (8), 
		\begin{align*}
			&(\sigma_\epsilon^2)^{(t+1)} = V_t + \frac{\sum_i \sum_j (y_{ij} - \beta_0^{(t)} - \beta_1^{(t)}x_{ij})^2 + J \sum_i E_t^2 - 2\sum_i \sum_j (y_{ij} - \beta_0^{(t)} - \beta_1^{(t)}x_{ij})E_t}{IJ}\\[10pt]
			&(\sigma_u^2)^{(t+1)} = V_t + \frac{J\sum_i E_t^2}{IJ}
		\end{align*}
		Repeat the above improvements until convergence.
	\subsection{}
		\lstinputlisting[caption=problem 5]{5.py}

\end{document}
























\documentclass{article}
\usepackage[margin = .7in]{geometry}
\usepackage[dvipdfmx]{graphicx}
\usepackage{listings}
\usepackage{amsmath}
\usepackage{bm}
\lstset{%
  language={python},
  basicstyle={\small},%
  identifierstyle={\small},%
  commentstyle={\small\itshape},%
  keywordstyle={\small\bfseries},%
  ndkeywordstyle={\small},%
  stringstyle={\small\ttfamily},
  frame={tb},
  breaklines=true,
  columns=[l]{fullflexible},%
  numbers=left,%
  xrightmargin=0zw,%
  xleftmargin=3zw,%
  numberstyle={\scriptsize},%
  stepnumber=1,
  numbersep=1zw,%
  lineskip=-0.5ex%
}

\begin{document}
\title{STAT6011/7611/6111/3317 \\ 
COMPUTATIONAL STATISTICS (2016 Fall) \\
Assignment 4}
\author{Kei Ikegami (u3535947)}
\maketitle

\section{}
The code is below. I denote the 'Fair' by 0 and 'Loaded' by 1 in this code.
	\lstinputlisting[caption=problem 1]{1.py}
And the results are as follows.
\begin{description}
	\item[(a)] [ 0,  0,  0,  0,  0,  0,  0,  0,  0,  0,  0,  0,  0,
      	0,  0,  0,  0,  0,  1,  1,  1,  1,  1,  1,  1,  1,
        0,  0,  1,  1,  1,  1,  0,  0,  1,  1,  1,  1,  1,
        0,  0,  0,  0,  0,  0,  0,  0,  0,  0,  0,  0,  0,
        0,  0,  0,  0,  0,  0,  0,  0,  0,  0,  0,  0,  0,
        0,  0,  0,  0]
	\item[(b)] [ 0,  0,  0,  0,  0,  0,  0,  0,  0,  0,  0,  0,  0,
        0,  0,  0,  0,  0,  0,  1,  1,  1,  1,  1,  1,  1,
        1,  1,  1,  1,  1,  1,  1,  1,  1,  1,  1,  1,  1,
        1,  0,  1,  0,  0,  0,  0,  0,  0,  0,  0,  0,  0,
        0,  0,  0,  0,  0,  0,  0,  0,  0,  0,  0,  0,  0,
        0,  0,  0,  0]
	\item[(c)] [ 0,  1,  0,  0,  0,  1,  0,  0,  0,  0,  0,  0,  0,
        1,  0,  0,  1,  1,  1,  1,  1,  1,  0,  1,  1,  0,
        0,  1,  1,  1,  1,  0,  0,  1,  1,  1,  1,  1,  0,
        0,  0,  1,  0,  0,  1,  1,  0,  0,  0,  0,  0,  0,
        0,  1,  0,  0,  0,  0,  1,  0,  0,  1,  0,  0,  0,
        1,  0,  0,  0]
	\item[(d)] [ 0,  1,  0,  0,  0,  1,  0,  0,  0,  0,  0,  0,  0,
        1,  0,  0,  1,  1,  1,  1,  1,  1,  0,  1,  1,  0,
        0,  1,  1,  1,  1,  0,  0,  1,  1,  1,  1,  1,  0,
        0,  0,  1,  0,  0,  1,  1,  0,  0,  0,  0,  0,  0,
        0,  1,  0,  0,  0,  0,  1,  0,  0,  1,  0,  0,  0,
        1,  0,  0,  0]
\end{description}
\section{}
	\lstinputlisting[caption=problem 2]{2.py}
	In the below question, I mean the weights by the values multiplied by weight functions.
	\subsection{}
	 	I prove the integral of standard normal probability distribution function is equal to $1$.
	\subsection{}
		I use the above code to compute the approximation of the integral.\par
		When the number of nodes is 5, \par the approximation result is 0.99684628222456162, \par the nodes are [-2.02018287, -0.95857246,  0.        ,  0.95857246,  2.02018287], \par the weights are [1.1814886255359833,
 0.98658099675142807,
 0.94530872048294179,
 0.98658099675142807,\par
 1.1814886255359833]
 		\par
		\ 
		\par
		When the number of nodes is 10, \par
		the approximation result is 0.99998763906433263 \par
		the nodes are [-3.43615912, -2.53273167, -1.75668365, -1.03661083, -0.34290133,
         0.34290133,  1.03661083,  1.75668365, \par 2.53273167,  3.43615912], \par
         	and the weights are [1.0254516913657519,
 0.82066612640481784,
 0.74144193194356545,
 0.70329632310490586,\par
 0.6870818539512733,
 0.6870818539512733,
 0.70329632310490586,
 0.74144193194356545,
 0.82066612640481784,\par
 1.0254516913657519] \par
 		\ 
		\par
 		When the number of nodes is 20, \par
		the approximation result is 0.99999999979803234, \par
		the nodes are [-5.38748089, -4.60368245, -3.94476404, -3.34785457, -2.78880606,
        -2.254974  , -1.73853771, -1.23407622, \par -0.73747373, -0.24534071,
         0.24534071,  0.73747373,  1.23407622,  1.73853771,  2.254974  ,
         2.78880606,  3.34785457,  \par 3.94476404,  4.60368245,  5.38748089], \par
         	and the weights are [0.89859196145317,
 0.70433296117692357,
 0.62227869619138665,
 0.57526244285250083,\par
 0.54485174236452072,
 0.52408035094855054,
 0.50967902711745705,
 0.49992087133628998,
 0.4938433852720529,\par
 0.49092150066674595,
 0.49092150066674595,
 0.4938433852720529,
 0.49992087133628998,
 0.50967902711745705,\par
 0.52408035094855054,
 0.54485174236452072,
 0.57526244285250083,
 0.62227869619138665,
 0.70433296117692357,\par
 0.89859196145317] \par
 		\ 
		\par
 		When the number of nodes is 30, \par
		the approximation result is 0.99999999999999634, \par
		the nodes are [-6.86334529, -6.13827922, -5.53314715, -4.98891897, -4.48305536,
        -4.0039086 , -3.54444387, -3.09997053, \par-2.66713212, -2.24339147,
        -1.82674114, -1.4155278 , -1.00833827, -0.60392106, -0.20112858,
         0.20112858,  0.60392106, \par 1.00833827,  1.4155278 ,  1.82674114,
         2.24339147,  2.66713212,  3.09997053,  3.54444387,  4.0039086 ,
         4.48305536,  \par 4.98891897,  5.53314715,  6.13827922,  6.86334529], \par
         and the weights are [0.83424747101269592,
 0.64909798155433118,
 0.56940269194957616,
 0.52252568933130883,\par
 0.49105799583287552,
 0.4683748125647248,
 0.451321035991189,
 0.43817702265268194,
 0.42791806293273793,\par
 0.41989500373682354,
 0.41367936361113872,
 0.40898157500353133,
 0.40560512332568432,
 0.40341981692480389,\par
 0.40234606670190304,
 0.40234606670190304,
 0.40341981692480389,
 0.40560512332568432,
 0.40898157500353133,\par
 0.41367936361113872,
 0.41989500373682354,
 0.42791806293273793,
 0.43817702265268194,
 0.451321035991189,\par
 0.4683748125647248,
 0.49105799583287552,
 0.52252568933130883,
 0.56940269194957616,
 0.64909798155433118,\par
 0.83424747101269592]. \par
         
\section{}
	\lstinputlisting[caption=problem 3]{3.py}
	In the below questions, I mean the weights by the values not multiplied by the weight functions.
	\subsection{}
		When I use Legendre polynomial as orthogonal polynomial, 
		\par
		the approximation result is -49.506283813990549,
		\par
		the nodes are [-0.90617985, -0.53846931,  0.        ,  0.53846931,  0.90617985],\par
		the weights are [ 0.23692689,  0.47862867,  0.56888889,  0.47862867,  0.23692689].
		\par
		\ 
		\par
		When I use Chebyshev type 1 polynomial as orthogonal polynomial, 
		\par
		the approximation result is -57.161045874594777,
		\par
		the nodes are [ -9.51056516e-01,  -5.87785252e-01,   6.12323400e-17,
          5.87785252e-01,   9.51056516e-01,]\par
		the weights are [ 0.62831853,  0.62831853,  0.62831853,  0.62831853,  0.62831853].
		\par
		\ 
		\par
		When I use Chebyshev type 2 polynomial as orthogonal polynomial, 
		\par
		the approximation result is -40.789633611622008,
		\par
		the nodes are [ -8.66025404e-01,  -5.00000000e-01,   6.12323400e-17,
          5.00000000e-01,   8.66025404e-01]\par
		the weights are [ 0.13089969,  0.39269908,  0.52359878,  0.39269908,  0.13089969].
		
	\subsection{}
		When I use Legendre polynomial as orthogonal polynomial, 
		\par
		the approximation result is -49.493963006199031,
		\par
		the nodes are [-0.97390653, -0.86506337, -0.67940957, -0.43339539, -0.14887434,
         0.14887434,  0.43339539,  0.67940957, \par 0.86506337,  0.97390653],\par
		the weights are [ 0.06667134,  0.14945135,  0.21908636,  0.26926672,  0.29552422,
         0.29552422,  0.26926672,  0.21908636, \par 0.14945135,  0.06667134].
		\par
		\ 
		\par
		When I use Chebyshev type 1 polynomial as orthogonal polynomial, 
		\par
		the approximation result is -50.987455239343021,
		\par
		the nodes are [-0.98768834, -0.89100652, -0.70710678, -0.4539905 , -0.15643447,
         0.15643447,  0.4539905 ,  0.70710678, \par 0.89100652,  0.98768834]\par
		the weights are [ 0.31415927,  0.31415927,  0.31415927,  0.31415927,  0.31415927,
         0.31415927,  0.31415927,  0.31415927, \par 0.31415927,  0.31415927].
		\par
		\ 
		\par
		When I use Chebyshev type 2 polynomial as orthogonal polynomial, 
		\par
		the approximation result is -47.101342989695162,
		\par
		the nodes are [-0.95949297, -0.84125353, -0.65486073, -0.41541501, -0.14231484,
         0.14231484,  0.41541501,  0.65486073, \par 0.84125353,  0.95949297]\par
		the weights are [ 0.02266894,  0.08347854,  0.16312218,  0.23631356,  0.27981494,
         0.27981494,  0.23631356,  0.16312218, \par 0.08347854,  0.02266894].
\section{}
	\lstinputlisting[caption=problem 4]{4.py}
	\subsection{}
	
	\subsection{}
	By using the above code, I get the estimation result after 100 iterations. \par The result is $[p_a, p_b, p_o] = [ 0.28021178,  0.28021178,  0.43957643]$.
\section{}
	\lstinputlisting[caption=problem 5]{5.py}

\end{document}























